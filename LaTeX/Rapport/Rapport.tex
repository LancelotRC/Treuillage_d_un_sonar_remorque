\documentclass[12pt,a4paper]{report}

%--------------------------------------------------------
% Encodage et langues
%--------------------------------------------------------
\usepackage[utf8]{inputenc}
\usepackage[T1]{fontenc}
\usepackage[french]{babel}

%--------------------------------------------------------
% Packages classiques
%--------------------------------------------------------
\usepackage{graphicx}
\usepackage{amsmath, amsfonts, amssymb}
\usepackage{geometry}
\usepackage{hyperref}
\usepackage{caption}
\usepackage{subcaption}
\usepackage{float}
\usepackage{enumitem}
\usepackage{fancyhdr}
\usepackage{xcolor}
\usepackage{ifthen}
\usepackage{tikz}
\usepackage[most]{tcolorbox}
\usepackage{pagecolor}
\usepackage{sectsty}
\usepackage{helvet}
\usepackage{setspace}
\usepackage[pages=some]{background}
\usepackage{amssymb}
\usetikzlibrary{arrows.meta, angles, quotes}


%--------------------------------------------------------
% Couleurs de la charte ENSTA
%--------------------------------------------------------
\definecolor{bleuENSTA}{HTML}{002A4B}
\definecolor{corailENSTA}{HTML}{F05A28}
\definecolor{grisENSTA}{gray}{0.3}

%--------------------------------------------------------
% Mise en page
%--------------------------------------------------------
\geometry{margin=2.5cm}
\renewcommand{\familydefault}{\sfdefault}
\setlength{\headheight}{40pt}
\pagestyle{fancy}
\fancyhf{}
\lhead{Projet UE 2.2 - Treuillage sonar remorqué}
\rhead{\thepage}
\sectionfont{\color{bleuENSTA!80}}
\subsectionfont{\color{bleuENSTA!70}}
\renewcommand{\headrulewidth}{1pt}
\renewcommand{\headrule}{\color{bleuENSTA!4}\hrule height 1pt}
\renewcommand{\footrulewidth}{0.5pt}
\renewcommand{\footrule}{\color{bleuENSTA!3}\hrule height 0.5pt}


%--------------------------------------------------------
% Début du document
%--------------------------------------------------------


\begin{document}

%--------------------------------------------------------
% Page de garde ENSTA
%--------------------------------------------------------


\backgroundsetup{
  scale=1,
  angle=0,
  opacity=0.5,
  contents={
    \begin{tikzpicture}[remember picture, overlay]
      \shade[inner color=bleuENSTA!60, outer color=white]
        (current page.north west) circle (20cm);
      \shade[inner color=bleuENSTA!60, outer color=white]
        (current page.south east) circle (25cm);
    \end{tikzpicture}
  }
}
\BgThispage

\begin{titlepage}

    \color{bleuENSTA}
    \centering
    \vspace*{1cm}



    {\Huge \bfseries Projet UE 2.2 \\[0.5cm]}
    {\LARGE Treuillage sonar remorqué}

    \vspace{2.5cm}
    \includegraphics[height = 3cm]{logoENSTA/PNG/horizontal/Ensta-logotype-IP-Paris-RVB-Horizontal-Bleu}
    \includegraphics[height = 3cm]{DeepVisionSonar}


    \vspace{2.5cm}
    \textbf{Élèves}
    Thomas FERRER \\
    Ewen PINSARD \\
    Elouan STASSEN \\
    Lancelot RAMIS


    \vspace{1.8cm}
    \textbf{Encadrants :} \\[0.3cm]
    Jean-François GUILLEMETTE \\
    Irvin PROBST

    \vfill
    {\large ENSTA Bretagne\\
    Institut Polytechnique de Paris}

    \vspace{1cm}
    {\today}

\end{titlepage}

\newpage
\tableofcontents
\vspace{5cm}
\newpage

\fancyhead[L]{%
  \includegraphics[height=1cm]{logoENSTA/PNG/horizontal/Ensta-logotype-IP-Paris-RVB-Horizontal-Bleu}
}

%----------------------------------------
\chapter{Introduction}
%----------------------------------------

Les sonars latéraux sont largement utilisés pour détecter la présence d’objets posés sur les fonds marins.
Ils sont montés sur des « poissons » remorqués, immergés à faible distance du fond afin de bénéficier d’une bonne rasance.
Ce mode de détection permet de repérer les objets par leur « ombre acoustique », formée par la réflexion des ondes sonores sur ces derniers.
\\ \\

Cependant, plusieurs défis techniques doivent être relevés pour une mise en œuvre efficace.
L’un des principaux est le positionnement du sonar : les données recueillies sont géo-référencées à partir de la
position GPS du navire, alors que le sonar est décalé horizontalement vers l’arrière. Il est donc nécessaire d’estimer
cette distance, appelée « layback », pour corriger le positionnement. Cela peut être réalisé soit expérimentalement,
soit par modélisation de la déformée du câble de remorquage.
\\ \\

Par ailleurs, la mise à l’eau du sonar repose actuellement sur un touret manuel, peu pratique à utiliser.
L’objectif du projet est donc double : modéliser la déformée du câble pour estimer le layback,
et concevoir un support pour un treuil motorisé, permettant de faciliter et d’automatiser la mise à la mer du
sonar.

%----------------------------------------

\chapter{Analyse du cahier des charges}
%----------------------------------------

\section{Objectifs fonctionnels}

Les objectifs principaux du projet sont les suivants :

\begin{itemize}

  \item Permettre la mise à l’eau et la récupération du sonar grâce à un treuil motorisé ;
  \item Estimer la déformée du câble afin de corriger le positionnement géographique des mesures du sonar
  (calcul du layback) ;
  \item Étudier les efforts mécaniques subis par le câble et le point d’ancrage ;
  \item Concevoir une structure de support pour le treuil adaptée à l’environnement du navire « Mélité » ;
  \item Proposer une solution éventuellement évolutive vers un système automatique de régulation de profondeur du sonar.

\end{itemize}

\section{Contraintes techniques}


Le cahier des charges impose plusieurs contraintes à respecter :

\begin{itemize}
  \item Le treuil doit permettre l’enroulement homogène de 50 m de câble électroporteur ;
  \item Il doit supporter une tension en pointe d’au moins 1500~N, voire 2000~N en cas d’accrochage du sonar ;
  \item La structure de support devra être compatible avec les platines latérales présentes sur la plage arrière de la
  « Mélité » ;
  \item Le support devra être suffisamment robuste et ne pas interférer avec les autres équipements du navire ;
  \item Le projet doit être mené en mode collaboratif, avec une démarche rigoureuse de conception
  (analyse fonctionnelle, modélisation, maquette numérique, choix des matériaux).

\end{itemize}


Une tâche complémentaire optionnelle consiste à concevoir un système de trancannage pour assurer un enroulement
optimal du câble:contentReference[oaicite:1]{index=1}.

\newpage

%----------------------------------------
% Étude théorique
%----------------------------------------

\chapter{Etude théorique}

\section{Modélisation du système}

Pour recueillir des données exploitables à partir du sonar, il est nécessaire de connaître sa position exacte.
En effet, si le navire tracteur peut être géolocalisé par GPS, ce n’est pas le cas du sonar, car les ondes électromagnétiques pénètrent très peu dans l’eau.
La difficulté réside donc dans l’estimation de la position du sonar immergé.

Une solution envisageable consiste à estimer la déformée du câble à partir des efforts subis par le système.
Cela permet ensuite d’estimer la distance horizontale entre le navire et le sonar, appelée layback, et ainsi d’en déduire une position approximative du sonar.

\subsection{Démarche}

\begin{itemize}


  \item Déterminer les efforts s'exerçant à l'extrémité du câble en isolant le sonar.
  \item Discrétiser le câble en $n$ éléments indéformables.
  \item Calculer l'inclinaison du premier segment de câble par rapport à la verticale, en utilisant l'équilibre des forces appliquées sur le sonar.
  \item Réaliser une itération sur l’ensemble des segments restants pour déterminer progressivement la forme du câble.
  \item Estimer la distance horizontale entre le sonar et le navire (layback) et verticale (profondeur) à partir de la déformée obtenue.

\end{itemize}

\subsection{Données, paramètres et hypothèses de travail}

\paragraph{Hypothèses de modélisation :}
\begin{itemize}
    \item Cable pesant, flottant et non élastique
    \item Angle du premier élement de cable dans l'axe de la résultante sonar <-> câble
    \item Cable orienté selon son effort
\end{itemize}

\paragraph{Données d’entrée :}
\begin{itemize}
  \item Vitesse du bateau : 4 nœuds, soit 2{,}05 m/s.
  \item Diamètre du câble : 6{,}3 mm.
  \item Longueur du câble : 50 m.
  \item Propriétés de l’eau de mer (à 15\,°C) :
  \begin{itemize}
    \item Masse volumique : \( 1025~\text{kg/m}^3 \).
    \item Viscosité dynamique : \( 1{,}22 \times 10^{-3}~\text{Pl} \).
  \end{itemize}
\end{itemize}


\begin{figure}[H]
\centering
\begin{tikzpicture}[scale=0.2]

    \coordinate (O) at (0,0);
    \coordinate (S) at (-39.2306, -27.8218);

    \fill[blue!10] (-50, 0) -- (10, 0) -- (10, -35) -- (-50, -35) -- cycle;

    % Déformée

    \draw [thick, orange]
    plot coordinates {
    (-39.2306, -27.8218)
    (-39.1764, -26.8233)
    (-39.0070, -25.8378)
    (-38.7318, -24.8764)
    (-38.3620, -23.9472)
    (-37.9127, -23.0539)
    (-37.3962, -22.1976)
    (-36.8248, -21.3769)
    (-36.2077, -20.5900)
    (-35.5530, -19.8341)
    (-34.8666, -19.1069)
    (-34.1538, -18.4056)
    (-33.4182, -17.7281)
    (-32.6633, -17.0723)
    (-31.8915, -16.4364)
    (-31.1052, -15.8186)
    (-30.3059, -15.2177)
    (-29.4952, -14.6321)
    (-28.6744, -14.0610)
    (-27.8445, -13.5031)
    (-27.0064, -12.9576)
    (-26.1608, -12.4237)
    (-25.3086, -11.9006)
    (-24.4501, -11.3876)
    (-23.5861, -10.8843)
    (-22.7168, -10.3899)
    (-21.8427, -9.9041)
    (-20.9643, -9.4263)
    (-20.0817, -8.9562)
    (-19.1953, -8.4933)
    (-18.3052, -8.0374)
    (-17.4119, -7.5880)
    (-16.5154, -7.1449)
    (-15.6160, -6.7078)
    (-14.7138, -6.2765)
    (-13.8090, -5.8506)
    (-12.9017, -5.4301)
    (-11.9921, -5.0146)
    (-11.0803, -4.6040)
    (-10.1664, -4.1981)
    (-9.2505, -3.7967)
    (-8.3327, -3.3997)
    (-7.4131, -3.0069)
    (-6.4917, -2.6181)
    (-5.5688, -2.2333)
    (-4.6442, -1.8522)
    (-3.7181, -1.4749)
    (-2.7906, -1.1011)
    (-1.8617, -0.7307)
    (-0.9315, -0.3637)
    (0.0000, 0.0000)
    };

    % Bateau
    \draw[fill=yellow!30] (0,-0.5) -- (6,-0.5) -- (8,1.5) -- (5, 1.3) -- (4.5, 3) -- (3, 3)
    -- (3.2, 1) -- (0, 1) -- cycle;

    % Sonar
    \draw[fill=gray!50]  (S) ellipse (1 and 0.5);

    % Surface de l'eau
    \draw[blue!50!white, thick] (-50, 0) -- (10, 0) ;

    \node at (-5, 2) {$La \ melitee$};
    \node at (-34, -27) {$Sonar$};
    \draw[red, thick, ->] (S) -- ++ (-1, -5) node[left] {$\vec{F}_{sonar}$};

\end{tikzpicture}
\caption{Vue générale : bateau, câble déformé et sonar}
\end{figure}

\subsection{Modélisation des efforts hydrodynamiques}

Pour chaque élément de câble, on cherche à définir l'angle suivant à partir des données déja connues.
On découpe alors deux parties :
\begin{itemize}
    \item L'initialisation, avec les efforts du sonar et une estimation de l'angle de la première section de câble
    \item La reccurence, on calcul l'angle de la section suivante en connaissant celui de la section actuelle et celui
    de l'effort de la section précédente.
\end{itemize}

\subsubsection*{Isolement du sonar}


L'initialisation dans un premier temps est plutot évidente, on connait les efforts du sonar sur le cable et donc les
efforts du cable sur le sonar par principe d'action réciproque.
Dans le référentiel terrestre supposé galiléen, nous supposons le sonar en translation rectiligne uniforme.
On considère la force de pesanteur et la poussée d'archimède (ces deux sont représentés par ${\vec{P}}$),
les frottements visqueux du sonar dans l'eau $\vec{f_{frot}}$, ainsi que la force du cable sur le sonar $\vec{F}_{cable}$

Dimension du sonar:
\begin{itemize}
    \item Longueur: 830 m
    \item Diamètre: 60 m
\end{itemize}

\begin{figure}[H]
\centering
\begin{tikzpicture}[scale=1.3, >=Stealth]

    % Sonar
    \draw[fill=gray!30] (0,0) ellipse (3 and 0.3);
    \node at (1.5, 0.8) {Sonar};

    \node at (0,0) {+} node [above right] {${G}$};
    % Force du câble
    \draw[->, blue] (0,0.3) -- ++(0.4, 2) node[above right] {$\vec{F}_{cable}$};
    % Poids
    \draw[->, red] (0,0) -- ++(0,-2) node[below right] {$\vec{P}$};
    % frottements
    \draw[->, red] (0,0) -- ++(-0.4, 0) node[left] {$\vec{f}_{frot}$};
    %Direction d'avancement
    \draw[->, very thick, black] (2.5,1) -- ++(1,0) node[above left] {$\vec{v}_{sonar}$};
    \draw[->, very thick, black] (-2.5,2) -- ++(0,-1) node[above right] {$\vec{g}$};

    \draw[thin, red, dashed] (0, -2) -- ++(-0.4, 0) -- ++(0,+2);
    \draw[->, very thick, red] (0, 0) -- ++(-0.4, -2) node[left] {$\vec{F}$};


\end{tikzpicture}
\caption{Efforts appliqués sur le sonar}
\end{figure}


Dans le référentiel terrestre, supposé galiléen, nous considérons le sonar en translation rectiligne uniforme.

Les forces appliquées sur le système (le sonar) sont les suivantes :
\begin{itemize}
  \item La force de pesanteur \( \vec{P} \).
  \item La poussée d’Archimède \( \vec{\Pi} \).
  \item La force de traînée hydrodynamique tangentielle.
  \item Les frottements dus à l’eau (traînée normale).
  \item La force exercée par le câble sur le sonar.
\end{itemize}

L’ensemble de ces forces est pris en compte dans l’analyse dynamique pour déterminer l’orientation initiale du câble
ainsi que le comportement du système tracté.

On en déduit l'équation suivante issue du principe fondamental de la statique
\\
\[
    \vec{F}_{cable} + \vec{f}_{frot} + \vec{P} = \vec{0} \\
    \Leftrightarrow  {F}_{cable} = - (\vec{f}_{frot} \ \vec{P})
\]


Cette équation permet ainsi de déduire l'orientation de l'effort du cable sur le sonar. \\
Attention, ici se joue une approximation importante, on sait que le cable sera orienté entre la verticale et la
direction de son effort. En effet, son inclinaison dépend des efforts qui s'appliquent en ses extrémités mais aussi
celles qui s'appliquent en son milieu (forces de trainée sur le câble). Ces efforts implique que le cable ne tire pas
dans son axe mais bien avec un angle dépendant des frottements visqueux.

\subsubsection*{Isolement du cable}

projection sur la normal du câble :
\[
    -F_{nc} + \sin(\alpha_1) \times F_1 - \sin(\alpha_0) \times (F_0 - P_{c}) = 0
\]

projection sur la tengente du câble :

\begin{align*}
    &-F_{t} + \cos(\alpha_1) \cdot F_1 - \cos(\alpha_0) \cdot (F_0 + P_c) = 0 \\
    &\hspace*{2cm}\left\{
    \begin{aligned}
        (\sin(\alpha_1) \cdot F_1)^2 &= \left(F_{n} + \sin(\alpha_0) \cdot (F_0 - P_c)\right)^2 \\
        (\cos(\alpha_1) \cdot F_1)^2 &= \left(F_{n} + \cos(\alpha_0) \cdot (F_0 + P_c)\right)^2
    \end{aligned}
    \right. \\
    &\left( \sin^2(\alpha_1) + \cos^2(\alpha_1) \right) \cdot F_1^2 =
    \left(F_{n} + \sin(\alpha_0) \cdot (F_0 - P_c)\right)^2 +
    \left(F_{n} + \cos(\alpha_0) \cdot (F_0 + P_c)\right)^2 \\
    &\Rightarrow F_1 = \sqrt{
        \left(F_{n} + \sin(\alpha_0) \cdot (F_0 - P_c)\right)^2 +
        \left(F_{n} + \cos(\alpha_0) \cdot (F_0 + P_c)\right)^2
    } \\
    &\alpha_1 = \arcsin\left(
        -\frac{1}{F_1} \cdot \left(F_{n} + \sin(\alpha_0) \cdot (F_0 - P_c)\right)
    \right)
\end{align*}



\begin{figure}[H]
\centering
\begin{tikzpicture}[scale=1.2, >=Stealth]

    \coordinate (A) at (-1, -2);
    \coordinate (B) at (1, 2);
    \coordinate (G) at (0, 0);
    \coordinate (Ab) at (-1.5, -3);
    \coordinate (Bb) at (1.5, 3);

        \draw[->, very thick, black] (2.5,1) -- ++(1,0) node[above left] {$\vec{v}_{sonar}$};
    \draw[->, very thick, black] (-2.5,2) -- ++(0,-1) node[above right] {$\vec{g}$};

    % Segment de câble
    \draw[thick] (A) -- (B);
    \draw[thin, dashed] (Ab) -- (Bb);
    \node at (1,0) {Section $i$};

    % Forces aux extrémités
    \draw[->, thick, blue] (A) -- ++(-0.3,-1.5) node[below left] {$\vec{F}_{i-1}$};
    \draw[->, thick, blue] (B) -- ++(0.8, 0.8) node[above right] {$\vec{F}_{i}$};

    % Poids et traînée
    \draw[->, red] (G) -- ++(0,-1) node[below] {$\vec{P}$};
    \draw[->, red] (G) -- ++(-2,0) node[left] {$\vec{f}$};

    % Angles

    % alpha_{i-2}
    \coordinate (C) at (-1, -1);
    \coordinate (D) at (-1.2,-3);
    \draw pic[draw = blue, ->, angle eccentricity = 1] {angle = C--A--D};
    \draw [thick, green!70!black, dashed] (C) -- ++ (0, -2);
    \node [blue] at (-1.8, -1.8) {$\alpha_{i-2}$};

    % alpha_{i}
    \coordinate (E) at (1, 3);
    \coordinate (F) at (0.2, 1.2);
    \draw pic[draw = blue, ->, angle eccentricity = 1] {angle = E--B--F};
    \draw [thick, green!70!black, dashed] (E) -- ++ (0, -2);
    \node [blue] at (0.2, 2) {$\alpha_{i}$};
    \draw [thick, blue, dashed] (B) -- (F);

    % alpha_{i-1}
    \coordinate (H) at (0, 1);

    \draw pic[draw = black, ->, angle eccentricity = 1] {angle = H--G--A};
    \draw [thick, green!70!black, dashed] (H) -- ++ (0, -1);
    \node [black] at (-0.7, 0.5) {$\alpha_{i-1}$};


\end{tikzpicture}
\caption{Bilan des forces sur un segement du câble}
\end{figure}


\subsection{Résultats et Validation du modèle}


Grâce au code python disponible en annexe, nous avons obtenus ces résulats:

Layback (à 4 nœuds): 39.97 m
Profondeur (à 4 nœuds): 26.82 m

Si les ordres de grandeurs semblent être respectés, on constate tout de même une différences de plusieurs mêtres,
allant jusqu'à près de 20\% dans les cas les plus critiques.
De tel écart pouvent sembler être mineur mais sont en réalité important lorsqu'on parle de récif dans un port par exemple.

On peut alors supposer plusieurs sources d'erreurs ou d'écarts:
\begin{itemize}
  \item Constantes d'entrée différentes et incertitude
  \item Calculs de trainées plus possibles
\end{itemize}


\section{Estimation de la position du sonar}
\begin{figure}[H]
    \begin{tikzpicture}

        \draw[ultra thick, ->] (0, 0) -- (10,0) node[below] {$\vec{x}$};
        \draw[ultra thick, ->] (0, 0) -- (0,7) node[left] {$\vec{y}$};


        \begin{scope}[rotate around = {-60:(2,0)}]
            \coordinate (S) at (0,1);
            \coordinate (B) at (0,5);
            \node at (S) {+};
            \node at (S) [below]{Sonar};
            \draw [fill = yellow!30] (B) -- +(0.3,0) -- +(0.5, 1) -- +(0, 1.8) -- +(-0.5,1) -- +(-0.3,0) -- cycle;
            \node at (B) {+};
            \node at (B) [below right]{Bateau};
            \draw[very thick] (B) -- (S);
            \draw[dashed] ([shift={(0.5,0)}]S) -- ([shift={(-1.5,0)}]S);
            \draw[dashed] ([shift={(0.5,0)}]B) -- ([shift={(-1.5,0)}]B);
            \draw[thick,->] ([shift={(-1,0)}]B) -- ([shift={(-1,0)}]S);
            \node at (-1.5,3) {layback};
        \end{scope}

    \end{tikzpicture}
\end{figure}




Une fois la valeur du \textit{layback} estimée, on peut calculer la position réelle du sonar tracté. Cela permet d’associer précisément les données acoustiques enregistrées par le sonar aux bonnes coordonnées géographiques, ce qui est essentiel pour produire des cartes fiables et interprétables.

Pour effectuer cette estimation, plusieurs données de navigation du navire et les caractéristiques du câble sont nécessaires, notamment :

\begin{itemize}
  \item Le \textbf{cap du navire} (ou direction), noté $\theta$ (en radians ou degrés, à convertir selon le contexte)
  \item Les \textbf{coordonnées du navire} à l’instant considéré, notées $x_{\text{navire}}$ et $y_{\text{navire}}$ (en système cartésien ou géographique)
  \item La \textbf{valeur du layback}, notée $L$, qui représente la distance entre le sonar et le point de remorquage à la poupe du navire, selon la modélisation du câble
\end{itemize}

En supposant une ligne droite entre le navire et le sonar (modèle simplifié), les \textbf{coordonnées du sonar} peuvent être estimées par les équations suivantes :

\begin{align*}
x_{\text{sonar}} &= x_{\text{navire}} - L \cdot \cos(\theta) \\
y_{\text{sonar}} &= y_{\text{navire}} - L \cdot \sin(\theta)
\end{align*}


Ce calcul permet de \textbf{géoréférencer correctement les données sonar}, ce qui est fondamental pour toute analyse bathymétrique, cartographique ou d'imagerie sous-marine.

%----------------------------------------
% Choix de conception
%----------------------------------------
\newpage


\section{Choix techniques et conception}

\subsection{Architecture retenue}
\subsection{Choix des composants}
\subsection{Calculs associés}



%----------------------------------------
% Simulation et validation
%----------------------------------------


\section{Simulation et validation}

\subsection{Hypothèses de simulation}
\subsection{Résultats obtenus}
\subsection{Analyse critique}


%----------------------------------------
% Bilan et perspectives
%----------------------------------------

\chapter{Conclusion et perspectives}

% Synthèse du projet, limites, pistes d’amélioration.


%----------------------------------------
% Annexes
%----------------------------------------


\appendix

\section{Annexe 1 : Données techniques}
\section{Annexe 2 : Codes ou algorithmes utilisés}

%----------------------------------------
% Bibliographie
%----------------------------------------

\begin{thebibliography}{9}

\bibitem{doc1}
Titre du document, Auteur, Année.


\bibitem{doc2}
Titre ou site, consulté le jj/mm/aaaa.


\end{thebibliography}

\end{document}
