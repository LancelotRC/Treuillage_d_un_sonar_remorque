\documentclass[a4paper,11pt]{article}
\usepackage[utf8]{inputenc}
\usepackage[T1]{fontenc}
\usepackage[french]{babel}
\usepackage{geometry}
\usepackage{enumitem}
\usepackage{xcolor}
\usepackage{array}
\usepackage{longtable}

\geometry{margin=2.5cm}

\title{Suivi des objectifs -- Projet Treuillage Sonar Remorqué 2025}
\date{}

\begin{document}

\maketitle

\section*{Tableau de suivi}

\begin{longtable}{|>{\raggedright\arraybackslash}p{7.5cm}|>{\centering\arraybackslash}p{3.5cm}|>{\raggedright\arraybackslash}p{3.5cm}|}
\hline
\textbf{Objectif} & \textbf{État d'avancement} & \textbf{Responsable(s)} \\
\hline
\multicolumn{2}{|l|}{\textbf{2. Exploitation des données recueillies}} & Thomas \\
\hline
Comprendre le fonctionnement du sonar latéral et interpréter les images acquises & avancé & Thomas\\
\hline
Associer les images du sonar à une localisation géographique précise & à travailler avec le layback & Ewen + Lancelot ? \\
\hline
Identifier les sources d’erreurs dans les mesures et les données collectées & Non commencé & Thomas ? \\
\hline

\multicolumn{2}{|l|}{\textbf{3. Modélisation de la déformée du câble de remorquage}} & Ewen + Lancelot\\
\hline
Modéliser la déformée du câble de remorquage (layback) & achevé & Ewen + Lancelot \\
\hline
Estimer la position réelle du sonar à partir des données de navigation du navire et du modèle de câble & à expliquer (ajouter layback en dir opp à la vitesse) & Ewen ? \\
\hline
Modéliser les efforts hydrodynamiques appliqués au câble et au sonar & achevé & Ewen + Lancelot \\
\hline
Valider la pertinence du modèle retenu (comparaison, sensibilité, simulations) & shéma python à interpréter & Lancelot + Ewen \\
\hline

\multicolumn{2}{|l|}{\textbf{4. Conception d’un support de treuil de mise à la mer}} & Elouan \\
\hline
Exprimer le besoin technique et les contraintes d’intégration sur le navire & à rédiger sur rapport & ? \\
\hline
Concevoir une solution mécanique de treuillage à installer sur le navire (support motorisé) & En cours ? & Elouan \\
\hline
Concevoir un tambour permettant l'enroulement et le déroulement efficaces du câble & En cours ? & Elouan \\
\hline
Étudier et concevoir un système de trancannage si nécessaire pour préserver le câble & Envisagé ? en cours ? & Elouan \\ % supprimer le texte incorrect
\hline
Garantir la protection du câble lors de sa mise à la mer & rouleaux de guidage ? & ? \\
\hline
Assurer la compatibilité de l'ensemble du système avec le navire La Mélité & En cours & Elouan \\
\hline
\end{longtable}

\section*{Légende}
\begin{itemize}[label=\textbullet]
  \item \textbf{Non commencé} : aucune tâche engagée
  \item \textbf{En cours} : des éléments en discussion ou conception
  \item \textbf{Terminé} : tâche finalisée et validée
\end{itemize}

\newpage

\section*{Détail des objectifs}

\subsection*{1. Comprendre le fonctionnement du sonar latéral et interpréter les images acquises}
Cette tâche consiste à :
\begin{itemize}
  \item Étudier le principe physique du sonar latéral (propagation, retour d’onde, angle d’émission, résolution).
  \item Se familiariser avec les types d’images produites (zones d’ombres, texture, intensité de retour).
  \item Identifier les objets caractéristiques sur les images (épaves, rochers, câbles, etc.).
\end{itemize}

\subsection*{2. Associer les images du sonar à une localisation géographique précise}
Cette tâche comprend :
\begin{itemize}
  \item L’analyse du format des données de navigation (GPS, cap, vitesse).
  \item Le calage spatial entre l’image acquise et la position du sonar.
  \item L’établissement d’une cartographie des images géolocalisées.
\end{itemize}

\subsection*{3. Modéliser la déformée du câble de remorquage (layback)}
Il s’agit ici de :
\begin{itemize}
  \item Étudier les modèles de catenaires et les lois de traction d’un câble dans l’eau.
  \item Prendre en compte les effets de traînée, de flottabilité et de vitesse du navire.
  \item Déduire la forme du câble et la position probable du sonar sous l’eau.
\end{itemize}

\subsection*{4. Estimer la position réelle du sonar à partir des données de navigation du navire}
Objectifs :
\begin{itemize}
  \item Intégrer les données de navigation avec le modèle du câble.
  \item Calculer le "layback" (décalage horizontal entre navire et sonar).
  \item Simuler ou valider cette estimation avec des cas tests.
\end{itemize}

\subsection*{5. Concevoir une solution mécanique de treuillage à installer sur le navire}
Il faut ici :
\begin{itemize}
  \item Identifier les contraintes mécaniques et spatiales sur le navire La Mélité.
  \item Concevoir une structure stable et résistante pour fixer le treuil.
  \item Choisir un moteur adapté au câble, à la tension, et aux efforts en jeu.
\end{itemize}

\subsection*{6. Concevoir un tambour permettant l'enroulement et le déroulement efficaces du câble}
Cette tâche vise à :
\begin{itemize}
  \item Définir le diamètre, la capacité, et la vitesse d’enroulement du tambour.
  \item Garantir un guidage régulier du câble.
  \item Prévoir la fixation au moteur ou à la transmission.
\end{itemize}

\subsection*{7. Étudier et concevoir un système de trancannage si nécessaire}
Le trancannage permet :
\begin{itemize}
  \item D’éviter l’accumulation désordonnée du câble sur le tambour.
  \item De guider le câble latéralement au fur et à mesure de son enroulement.
  \item Il peut être mécanique, motorisé ou synchronisé avec la rotation.
\end{itemize}

\subsection*{8. Garantir la protection du câble lors de sa mise à la mer}
Objectifs :
\begin{itemize}
  \item Étudier les risques d’usure ou d’arrachement du câble.
  \item Concevoir un guide-câble ou une rampe de lancement.
  \item Minimiser les chocs et les frictions sur le bateau.
\end{itemize}

\subsection*{9. Assurer la compatibilité du système avec le navire La Mélité}
Il faudra :
\begin{itemize}
  \item Relever les dimensions, emplacements disponibles et interfaces du navire.
  \item Vérifier la tenue mécanique du système.
  \item S’assurer que le fonctionnement n’interfère pas avec les manœuvres marines.
\end{itemize}

\end{document}
